\documentclass[12pt,a4paper]{article}
\usepackage{graphicx}
\usepackage{biblatex}
\usepackage{parskip}
\usepackage{listings}


\lstset{%
basicstyle=\ttfamily,
breaklines = true,
tabsize=2
}
\graphicspath{ {./Images/} }

\addbibresource{biblography.bib}
\setlength{\parskip}{1em}
\begin{document}
\begin{titlepage}
	\centering
	{\scshape\LARGE ELEC40006 \par}
	\vspace{1cm}
	{\scshape\Large Electronics design project\par}
	\vspace{1.5cm}
	{\huge\bfseries Circuit Simulation Report\par}
	\vspace{2cm}
	{\Large\ Adam Rehman\\ Brandon Cann\\ Xin Wang \par}
	\vfill
% Bottom of the page
	{\large \today\par}
\end{titlepage}

\tableofcontents
\pagebreak

\begin{abstract}
This report describes the design and implementation of a program that is capable of performing a transient simulation
by calculating the node voltages at each successive instant in time. This program parses the netlist file
into a graph data structure, performs analysis using conductance matrices and outputs the results in a CSV format.

-- How accurate is it?
\par
-- Comaparison to commercial software?
\end{abstract}
\pagebreak

\section{Overview of the report}
This report is the distillation of multiple research documents relating to different components of the program. \par
Section 2 gives an abstract view of the design of the program, breaking the program down into 3 modules.
Section 3 provides a summary of the testing methodologies and a comparison to both handwritten results and results of 
established circuit simulator software. 
Section 4 delves into the further work done and some potential ideas to build on.
Section 5, the last section, summarises the report and discusses our overall experiences with the development of this project.
\par
\textit{Talk about added functions and anything else.}
\pagebreak

\section{Project management}
\pagebreak

\subsection{Project timeline}

\pagebreak

\subsection{Management approach}
\pagebreak

\subsection{Project responsibilities breakdown}
\pagebreak

\section{Design of software modules}
\pagebreak

\subsection{Parser module}
\pagebreak
\subsubsection{Features}
\pagebreak
\subsubsection{Pseudocode}
\pagebreak

\subsection{Analysis module}
\pagebreak
\subsubsection{Features}
\pagebreak
\subsubsection{Pseudocode}
\pagebreak

\subsection{Transient module}
\pagebreak
\subsubsection{Features}
\pagebreak
\subsubsection{Pseudocode}
\pagebreak

\section{Testing}
\subsection{\textit{Data struct}}
The script \textit{Data struct test.sh}, when called, will compile \textit{Data struct.cpp} and passes in input 
text file \textit{Data struct input.txt}. 
\par
\textbf{Pictures}
\par
This test is used to check the format of CirElement data structure functions as envisioned and that the methods associated with
CirElement such as \textit{custom pow} functions correctly.
\pagebreak

\section{Optimisations}
Typical features like clearing vectors once it's function is done such as the tokensier
\pagebreak

\section{Software comparisons}
\pagebreak

\section{After project report}
\pagebreak

\printbibliography[title={References}]
\end{document}