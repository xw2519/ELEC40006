\documentclass[a4paper, titlepage]{article}

\usepackage{biblatex}
\addbibresource{biblography.bib}
\setcounter{tocdepth}{4}

\title{NetList Documentation}
\author{Adam, Brandon Cann, Xin Wang \thanks{document compiled by Xin Wang}}
\date{7 May 2020}

\setlength{\parskip}{1em}

\begin{document}

    \section{Adjacency List}
    An Adjacency List is a vector where each element is a pair, the pair being the edge destination and weight.

    \subsection{Adjacency Matrix}
    An Adjacency Matrix is a square matrix of size $N x N$. If the inputted graph has some edges from \textit{a} to \textit{b} then
    in the Adjacency Matrix at the $a^{th}$ row and $b^{th}$ will be non-zero for a weighted graph otherwise it will default to 0.

    \subsection{Trade-offs}
    \begin{itemize}
        \item For a sparse graph, Adjacency List requires less space because they do not waste any space to represent edges that are not present.
        \item Different data structure facilitate different operations. For example, finding all adjacent vertices in an Adjacency List is 
        found by reading the list. In an Adjacency Matrix, an entire row must be scanned.
    \end{itemize}
\end{document}